\chapter{Introduction}

This manual provides a reference for SuperBASIC, a beginner-friendly programming language for the Foenix F256 series of personal computers. It is not a tutorial and assumes the reader has some prior knowledge of programming in general and of BASIC in particular.\\

SuperBASIC builds upon the heritage of classic BASIC dialects while introducing a wide range of enhancements that make programming even more fun and accessible. It is tightly integrated with the advanced hardware capabilities of the F256 platform, offering first-class access to graphics, sprites, game controllers, sound, and more. Notable language additions include structured control flow, named subroutines, inline assembly, and an enriched standard library. \\

We’ve worked hard to make SuperBASIC a powerful and approachable language for curious minds of all ages. We hope you enjoy using it as much as we enjoyed creating it!

\section{Storage}

Programs are stored in ASCII format, so in cross-development, any editor can be used. LOAD, VERIFY, and SAVE read and write files in this format. Internally, the format is quite different.

\section{Memory usage}

It currently occupies 4 pages (32k) of Flash mapped into 2 pages (16k) of the 65C02 memory space, from \code{\$8000} to \code{\$BFFF}.\\

Memory usage is split into 2 parts. \\

The main space from \$2000-\$7FFF contains the tokenised BASIC code. Keywords such as REPEAT are replaced by a single byte, or for less common options, by two bytes.\\ 

Identifiers are replaced by a reference into the identifier table from \$1000-\$1FFF. The first part of this table is a list of identifiers along with the current value.\\

Arrays and allocated memory (using ALLOC) follow that.\\

Finally, string memory occupies the top of this memory area and works down.\\

This should be entirely transparent to the developer.

\section{Memory usage elsewhere}

SuperBASIC uses memory locations outside the normal 6502 address space as well. \\

If you use a bitmap, it will be placed at \$10000 in physical space and occupy 320x240 bytes.\\

If you use sprites, they are loaded to \$30000 in physical space, and the size depends on how many you have.\\

If you use tiles, the tile map is stored at \$24000 and the tile image data at \$26000. \\

Finally, if you cross-develop, the memory location from \$28000 onwards is used to store the BASIC code you have uploaded.\\

If you do not use any of these features directly (you can set up your own bitmaps and sprites yourself, and enter programs through the keyboard, saving to the SD Card or IEC drive) then the memory is all yours.
