\chapter{Getting started}

\section {Writing programs}
Programs in SuperBASIC are written in the 'classic' style, using line numbers. A line number on its own deletes a line.\\

\kwd{list} operates as in most other systems, except there is the option to \code{list <procedure>()}, which lists the given procedure by name. \kwd{list} also uses commas, not dashes, as some BASICs do, e.g. \code{list 100,300}.\\

It is easy to cross-develop in SuperBASIC (see Chapter~\ref{chap:crossdev}), writing a program on your favourite text editor, and transfering it to the F256 over USB using the FnxMgr or the Foenix IDE. It is also possible to develop without line numbers and have them added as the last stage before uploading.\\

Upper and lower case are considered to be the same, so variables \code{myName}, \code{MYNAME}, and \code{MyName} are all the same variable. The only place where case is specifically differentiated is in string constants.\\

Programs can be loaded or saved to SD Card or to an IEC type drive (the 5 pin DIN serial port) using the \kwd{save} and \kwd{load} commands.\\

The \path{documents} directory in the SuperBASIC GitHub repository has a simple syntax highlighter for the Sublime Text editor.\\
