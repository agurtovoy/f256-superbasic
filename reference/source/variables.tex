\chapter{Identifiers, Variables and Typing}
\label{chap:variables}

\section{Procedure and variable naming}

SuperBASIC lets you give long, descriptive names to both variables and procedures (also known as subroutines). Using clear names makes your programs easier to read and understand.\\

A name must start with a letter or an underscore (\code{\_}), and it can continue with any combination of letters, numbers, or underscores. A name may also end with a type character (\code{\$} or \code{\#}), which shows the kind of data stored in the variable (see the next section for more details).\\

Here are some examples of valid names:

\begin{verbatim}
    n
    count17
    number_of_lives
    str2num
    name$
    average#
\end{verbatim}

Here are some examples of \textit{invalid} names: 

\begin{lst}
    2string     ' cannot start with a number
    $name       ' $ character must be at the end
    total#sum   ' # character must be at the end
\end{lst}

\begin{implnote}
When a program is stored in memory, each identifier name is stored only once, regardless of how many times it appears in the program. Thus, aside from that single instance, using shorter identifier names provides no additional space savings.
\end{implnote}

\section{Types}

SuperBASIC supports three variable types: integers, floating-point numbers, and strings.\\

By default, a variable is an integer. Integers are whole numbers, such as \(-5\), \(0\), or \(4200\).  In SuperBASIC, they can go up to about \(2\) billion or down to \(-2\) billion.\\

A variable name ending with \code{\#} represents a floating-point number (a decimal). Floating-point numbers let you use very large, very small, or fractional values (like \(178.2\)). They are more flexible than integers, but sometimes less exact and a little slower to calculate.\\

A variable name ending with \code{\$} represents a string. A string is simply arbitrary text, up to 253 characters long. In program code, string values—called \emph{string literals}—are written inside quotation marks. For example, \code{"Arthur Dent"} is a string literal.\\

Here are some examples:

\begin{lst}
  100 count   = 19
  110 height# = 178.2
  120 name$   = "Arthur Dent"
\end{lst}

\section{Arrays}

An array is a collection of related variables that share the same name and are stored together. Arrays are created using the \kwd{dim} statement, followed by the array name and the number of elements. Each individual element is accessed by giving its index inside parentheses. Array indexes always begin at zero:

\begin{lst}
100 dim fruits$(3)
110 fruits$(0) = "apple"
120 fruits$(1) = "orange"
130 fruits$(2) = "banana"
140 print fruits$(0)  ' prints "apple"
150 print fruits$(2)  ' prints "banana"
\end{lst}

SuperBASIC supports both one-dimensional and two-dimensional arrays, with up to 254 elements in each dimension.\\

When first created, string array elements are empty strings, and number array elements are set to zero.\\ 

Here is a two-dimensional array of numbers, which you can think of like a grid with rows and columns:

\begin{lst}
100 dim grid(8,8)    ' 8 by 8 grid of numbers
110 grid(4,3) = 17
120 print grid(4,3)  ' prints 17
130 print grid(7,7)  ' prints 0
\end{lst}

