\chapter{Tiles and Tile Maps}

\section{Introduction}

SuperBASIC supports a single tile map, made up of 8x8 pixel images. A tile map can be up to 255 x 255 tiles in size.

\section{Setting up a Tile Map}

The TILES command sets up a tile map. For example, the following command sets up a 48x48 tile map at the default locations (see below), and turns it on.

\begin{verbatim}
	100 tiles dim 48,48 on
\end{verbatim}

\section{Manipulating a Tile Map}

The TILE command is used to manipulate a tile map, by either scrolling its position, or by writing to it.  These commands can be chained in a similar manner to the graphics drawing ones (so these two commands could be joined into one).\\

This example sets the draw pointer at tile 4 across, 5 down and draws the following tiles, writing horizontally, of tile 10, 3 tile 11s, and another tile 10. So it is not difficult to create maps programmatically.\\

Tile TO scrolls the tile map on the screen, this is in pixels not whole tiles.\\

The TILE() function reads the tile at the current map position(which following the code at line 100, should be 11).

\begin{verbatim}
	100 tile at 4,5 plot 10,11 line 3,10
	110 tile to 14,12
	120 t = tile(5,5)
\end{verbatim}
		
\section{Data formats}

There are two data files required for a tile map. One is the images file, which is a sequence of 64 bytes, representing an 8x8 tile. These are indexed from zero. This images file is held at \$26000 by default (though it can be placed anywhere). \\

The second is the map file itself. This could be created in some sort of editor, and loaded manually, or could be generated randomly. This is a word sequence, which is (map width x map height x 2) bytes in size, as specified in the Hardware reference manual. This map is held at \$24000 by default (it too can be placed anywhere).